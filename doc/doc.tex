\documentclass[12pt, a4paper]{article}
\usepackage[MeX]{polski} % Wsparcie dla PL
\usepackage[utf8]{inputenc} % Kodowanie UTF8 
\usepackage{geometry}
\usepackage{hyperref}
\usepackage{amsmath}
\title{\textbf{Implementacja algorytmu badającego podobieństwo trzech sekwencji}}
\author{Anna Stępień \\ Marek Lewandowski}
\date{}
\setlength{\parindent}{0in}

\begin{document}
\maketitle

\section{Zadanie}
Celem zadania jest zrealizowanie aplikacji, która oblicza najlepsze dopasowanie dla trzech sekwencji. 

Do realizacji zadania zostanie wykorzystany algorytm Needlemana -- Wunscha, którego uogólnienie dla trzech sekwencji można przedstawić następująco:

\begin{equation}
  F(i, j, k)= max \begin{cases}
    F(i-1, j-1, k-1) + e(s_i, t_j) + e(s_i, u_k) + e(t_j, u_k)\\
    F(i-1, j-1, k) + e(s_i, t_j) + d\\
    F(i-1, j, k-1) + e(s_i, u_k) + d\\
    F(i, j-1, k-1) + e(t_j, u_k) + d\\
    F(i-1, j, k) + 2d\\
    F(i, j-1, k) + 2d\\
    F(i, j, k-1) + 2d
  \end{cases}
\end{equation}

gdzie:\\
$e(x, y)$ - wartość macierzy kar i nagród\\
$d$ - kara za przerwę

\section{Założenia}

\subsection{Założenia dla algorytmu}
\begin{itemize}
\item sekwencje wejściowe są homologiczne,
\item sekwencje są sekwencjami DNA,
\item ocena dopasowania sekwencji oparta o punktację z macierzy PAM30 lub PAM70.
\end

\subsection{Założenia implementacyjne}
\begin{itemize}
\item interakcja z aplikacją typowo Unixowa - linia poleceń i pliki,
\item sekwencje wejściowe dostarczone w plikach,
\item format sekwencji - na początek wsparcie tylko dla jawnego tekstu,
\item implementacja aplikacji w języku Scala \href{http://www.scala-lang.org/}{Scala}
\end

\end{document}
