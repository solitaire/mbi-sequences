\documentclass[12pt, a4paper]{article}
\usepackage{polski}
\usepackage[utf8]{inputenc}
\usepackage[polish]{babel} 
\usepackage{geometry}
\usepackage{hyperref}
\usepackage{amsmath}
\usepackage[numbers]{natbib}
\title{\textbf{Implementacja algorytmu badającego podobieństwo trzech sekwencji}}
\author{Anna Stępień \\ Marek Lewandowski}
\date{}
\setlength{\parindent}{0in}

\begin{document}
\maketitle

\section{Zadanie}
Celem zadania jest zrealizowanie aplikacji, która oblicza najlepsze dopasowanie dla trzech sekwencji. 

Do realizacji zadania zostanie wykorzystany algorytm Needlemana -- Wunscha, którego uogólnienie dla trzech sekwencji można przedstawić następująco:

\begin{equation}
  F(i, j, k)= max \begin{cases}
    F(i-1, j-1, k-1) + e(s_i, t_j, u_k)\\
    F(i-1, j-1, k) + e(s_i, t_j, -)\\
    F(i-1, j, k-1) + e(s_i, -,u_k)\\
    F(i, j-1, k-1) + e(-, t_j, u_k)\\
    F(i-1, j, k) + e(s_i, -, -)\\
    F(i, j-1, k) + e(-, t_j, -)\\
    F(i, j, k-1) + e(-, -, u_k)
  \end{cases}
\end{equation}

gdzie:\\
macierz kar i nagród uwzględnia przerwy\\
$e(s_i, t_j, u_k) = e(s_i, t_j) + e(s_i, u_k) + e(t_j, u_k)$\\
\section{Założenia}

\subsection{Założenia dla algorytmu}
\begin{itemize}
\item sekwencje wejściowe są sekwencjami DNA (alfabet $\{A, G, C, T\}$),
\item wejściem jest plik z sekwencjami oraz plik z macierzą kar i nagród (uwzględniającą przerwy),
\item wyjściem jest plik tekstowy z najlepszym dopasowaniem i jego punktacją.
\end{itemize}

\subsection{Założenia implementacyjne}
\begin{itemize}
\item interakcja z aplikacją typowo Unixowa - linia poleceń i pliki,
\item sekwencje wejściowe dostarczone w plikach,
\item format sekwencji - na początek wsparcie tylko dla jawnego tekstu, kolejne sekwencje oddzielone znakiem nowej linii,
\item implementacja aplikacji w języku \href{http://www.scala-lang.org/}{Scala}
\end{itemize}

\nocite{*}
\bibliographystyle{plainnat}
\bibliography{bibliography}
\end{document}
