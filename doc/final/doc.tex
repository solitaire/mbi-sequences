\documentclass[12pt, a4paper]{article}
\usepackage{polski}
\usepackage[utf8]{inputenc}
\usepackage[polish]{babel} 
\usepackage{geometry}
\usepackage{hyperref}
\usepackage{amsmath}
\usepackage[numbers]{natbib}
\title{\textbf{Implementacja algorytmu badającego podobieństwo trzech sekwencji}}
\author{Anna Stępień \\ Marek Lewandowski}
\date{}
\setlength{\parindent}{0in}

\begin{document}
\maketitle

\section{Zadanie}
Celem zadania jest zrealizowanie aplikacji, która oblicza najlepsze dopasowanie dla trzech sekwencji. 

Do realizacji zadania zostanie wykorzystany algorytm Needlemana -- Wunscha, którego uogólnienie dla trzech sekwencji można przedstawić następująco:

\begin{equation}
  F(i, j, k)= max \begin{cases}
    F(i-1, j-1, k-1) + e(s_i, t_j, u_k)\\
    F(i-1, j-1, k) + e(s_i, t_j, -)\\
    F(i-1, j, k-1) + e(s_i, -,u_k)\\
    F(i, j-1, k-1) + e(-, t_j, u_k)\\
    F(i-1, j, k) + e(s_i, -, -)\\
    F(i, j-1, k) + e(-, t_j, -)\\
    F(i, j, k-1) + e(-, -, u_k)
  \end{cases}
\end{equation}

gdzie:\\
macierz kar i nagród uwzględnia przerwy\\
$e(s_i, t_j, u_k) = e(s_i, t_j) + e(s_i, u_k) + e(t_j, u_k)$\\

\section{Opis aplikacji}
Aplikacja została zrealizowana w~języku \href{http://www.scala-lang.org/}{Scala}. W~stosunku do założeń przedstawionych w~dokumentacji wstępnej, wprowadziliśmy jedną zmianę związaną ze sposobem wczytywania sekwencji. Jeden plik wejściowy, który zawierał trzy sekwencje oddzielone znakiem nowej linii, został zastąpiony poprzez osobne pliki dla każdej z~sekwencji. Format ten jest bardziej przyjazny dla użytkownika -- nie jest konieczne tworzenie specjalnych plików wejściowych. Ponadto wprowadziliśmy drugi obsługiwany format danych -- format FASTA. 

W~ramach projektu zrealizowaliśmy dwie wersje algorytmu Needlemana-Wunscha -- iteracyjną oraz rekurencyjną.

	\subsection{Wejście}
	Aplikacja na wejściu przyjmuje 3 ścieżki do plików z~sekwencjami (są one określone przez flagę \texttt{-seq}) oraz ścieżkę do pliku z~macierzą podobieństwa (określoną przez parametr \texttt{-sm}). Domyślnie uruchamiana jest iteracyjna wersja algorytmu. Wersja rekurencyjna uruchamiana jest po podaniu flagi \texttt{-r}.
	Przykład uruchomienia aplikacji: \\
	\texttt{scala mbi-sequences\_2.10-1.0.jar -seq s1 s2 s3 -sm similarity-matrix}
	\subsection{Wyjście}
	Wyjściem aplikacji jest najlepsze dopasowanie oraz jego koszt. Domyślnie dane te wypisywane są na standardowe wyjście -- po przekazaniu nazwy pliku możliwy jest zapisanie otrzymanych wyników.
	
\section{Testy}

\section{Wnioski}

\nocite{*}
\bibliographystyle{plainnat}
\bibliography{bibliography}
\end{document}
