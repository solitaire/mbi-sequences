\documentclass[12pt, a4paper]{article}
\usepackage{polski}
\usepackage[utf8]{inputenc}
\usepackage[polish]{babel} 
\usepackage{geometry}
\usepackage{hyperref}
\usepackage{amsmath}
\usepackage[numbers]{natbib}
\usepackage[section]{placeins}
\title{\textbf{Implementacja algorytmu badającego podobieństwo trzech sekwencji}}
\author{Anna Stępień \\ Marek Lewandowski}
\date{}
\setlength{\parindent}{0in}

\begin{document}
\maketitle

\section{Zadanie}
Celem zadania jest zrealizowanie aplikacji, która oblicza najlepsze dopasowanie dla trzech sekwencji. 

Do realizacji zadania zostanie wykorzystany algorytm Needlemana -- Wunscha, którego uogólnienie dla trzech sekwencji można przedstawić następująco:

\begin{equation}
  F(i, j, k)= max \begin{cases}
    F(i-1, j-1, k-1) + e(s_i, t_j, u_k)\\
    F(i-1, j-1, k) + e(s_i, t_j, -)\\
    F(i-1, j, k-1) + e(s_i, -,u_k)\\
    F(i, j-1, k-1) + e(-, t_j, u_k)\\
    F(i-1, j, k) + e(s_i, -, -)\\
    F(i, j-1, k) + e(-, t_j, -)\\
    F(i, j, k-1) + e(-, -, u_k)
  \end{cases}
\end{equation}

gdzie:\\
macierz kar i nagród uwzględnia przerwy\\
$e(s_i, t_j, u_k) = e(s_i, t_j) + e(s_i, u_k) + e(t_j, u_k)$\\

\section{Opis aplikacji}
Aplikacja została zrealizowana w~języku \href{http://www.scala-lang.org/}{Scala}. W~stosunku do założeń przedstawionych w~dokumentacji wstępnej, wprowadziliśmy jedną zmianę związaną ze sposobem wczytywania sekwencji. Jeden plik wejściowy, który zawierał trzy sekwencje oddzielone znakiem nowej linii, został zastąpiony poprzez osobne pliki dla każdej z~sekwencji. Format ten jest bardziej przyjazny dla użytkownika -- nie jest konieczne tworzenie specjalnych plików wejściowych. Ponadto wprowadziliśmy drugi obsługiwany format danych -- format FASTA. 

W~ramach projektu zrealizowaliśmy dwie wersje algorytmu Needlemana-Wunscha -- iteracyjną oraz rekurencyjną.

	\subsection{Wejście}
	Aplikacja na wejściu przyjmuje 3 ścieżki do plików z~sekwencjami (są one określone przez flagę \texttt{-seq}) oraz ścieżkę do pliku z~macierzą podobieństwa (określoną przez parametr \texttt{-sm}). Domyślnie uruchamiana jest iteracyjna wersja algorytmu. Wersja rekurencyjna uruchamiana jest po podaniu flagi \texttt{-r}.
	Przykład uruchomienia aplikacji: \\
	\texttt{scala mbi-sequences\_2.10-1.0.jar -seq s1 s2 s3 -sm similarity-matrix}
	\subsection{Wyjście}
	Wyjściem aplikacji jest najlepsze dopasowanie oraz jego koszt. Domyślnie dane te wypisywane są na standardowe wyjście.
	
\section{Testy}
W ramach testowania zaimplementowanego algorytmu przeprowadzone zostały dwa rodzaje testów: testy jednostkowe oraz testy funkcjonalne.
	\subsection{Testy jednostkowe}
	Testy jednostkowe miały na celu zweryfikowanie poprawności implementacji algorytmu. Przede wszystkim zweryfikowana została poprawność następujących funkcji programu:
	\begin{itemize}
		\item wczytywanie sekwencji z plików w prostym formacie oraz formacie FASTA,
		\item wczytywanie macierzy podobieństwa z pliku,
		\item działanie wersji iteracyjnej algorytmu,
		\item działanie wersji rekurencyjnej algorytmu,
		\item poprawność zwracanego kosztu dopasowania.
	\end{itemize}
	
	\subsection{Testy funkcjonalne}
	Podczas testów funkcjonalnych aplikacji zweryfikowane zostało działanie aplikacji dla rzeczywistych sekwencji pozyskanych ze strony \footnote{http://www.ebi.ac.uk}.
	Na potrzeby testów zostały utworzone dwa zbiory danych zawierające sekwencje o zróżnicowanej długości (długości wahały się od ok. 20 do ok. 100) -- jeden zbiór zawierał sekwencje w prostym formacie, natomiast drugi zawierał te same sekwencje w formacie FASTA. Dla każdego przypadku testowego została wybrana jedna sekwencja, na podstawie której, po dokonaniu modyfikacji (usunięciu losowych elementów oraz wprowadzeniu punkowych mutacji) utworzone zostały trzy sekwencje.
	
	Dla tak przygotowanych zestawów danych, kolejno przetestowane zostały następujące przypadki:
	\begin{itemize}
		\item działanie iteracyjnej wersji algorytmu dla sekwencji w prostym formacie,
		\item działanie iteracyjnej wersji algorytmu dla sekwencji w formacie FASTA,
		\item działanie rekurencyjnej wersji algorytmu dla sekwencji w prostym formacie,
		\item działanie rekurencyjnej wersji algorytmu dla sekwencji w formacie FASTA.
	\end{itemize}
	Następnie porównano otrzymane wyniki dla każdego z przypadków. Wyniki otrzymane dla obu formatów, zarówno przez algorytm w wersji iteracyjnej jak i rekurencyjnej, były jednakowe.

\section{Podsumowanie}
W ramach projektu zaimplementowane zostały dwie wersje algorytmu Needlemana-Wunscha badające podobieństwo trzech sekwencji. Zrealizowane zostały założenia przedstawione w dokumentacji wstępnej, a także rozszerzono aplikację o dodatkowy format danych. Przeprowadzone testy wykazały prawidłowe działanie stworzonej aplikacji. 

\nocite{*}
\bibliographystyle{plainnat}
\bibliography{bibliography}
\end{document}
